\documentclass[12pt,a4paper]{article}

\usepackage[T1]{fontenc}
\usepackage[utf8]{inputenc}
\usepackage[margin=2.5cm]{geometry}
\usepackage{graphicx}
\usepackage[hidelinks]{hyperref}
\usepackage{fancyhdr}
\usepackage{lastpage}
\usepackage{appendix}
\usepackage{color}
\usepackage{palatino}
\usepackage{changepage}
\usepackage{subcaption}
\usepackage{enumitem}
\usepackage{csquotes}
\usepackage{verbatim}
\usepackage[cache=false]{minted}

\usemintedstyle{fruity}

\title{Community detection in networks}
\author{Carlos Requena López}

%% Fancy layout
\pagestyle{fancy}
\lhead{INFO-F521}
\chead{}
\rhead{Carlos Requena}
\lfoot{}
\cfoot{}
\rfoot{Page \thepage\ of \pageref{LastPage}}
\renewcommand{\headrulewidth}{0.4pt}
\renewcommand{\footrulewidth}{0.4pt}


%%% --- %%% --- DOCUMENT START --- %%% --- %%%
\begin{document}
\thispagestyle{fancy}
\maketitle
\thispagestyle{fancy}

\section{Introduction}

The topic of this assignment is community detection in networks by
means of hierarchical clustering. Informally, communities can be seen
as dense (highly connected) subgraphs.

Hierarchical clustering (HC) refers to the technique of grouping
together elements that share some sort of similarity and separating
those that do not.

\section{Analysis and expectations}

Let $ G = (V, E) $ be an undirected graph.

$$ \mu_i = \frac{1}{n} \sum_{j=1}^{n} A_{ij} $$

$$ \sigma^2 = \frac{1}{n} \sum_{j=1}^{n} (A_{ij} - \mu_i)^2 $$

$$ x_{ij} = \frac{\frac{1}{n}\sum_{k=1}^{n}(A_{ik} - \mu_i)(A_{jk}-\mu_j)}{\sigma_i\sigma_j} $$

\section{Implementation}

The graph $ G $ is given as an adjacency matrix, which is parsed

\section{Results}

\section{Conclusion}

% \begin{figure}[ht!]
%   \centering
%   \includegraphics[width=1\textwidth]{some_graphic.png}
%   \caption{}
%   \label{fig:fig1}
% \end{figure}

\bibliographystyle{ieeetr}
\bibliography{main}
\nocite{*}

\appendix
\section{Appendix - code listing}

\inputminted{python}{../src/main.py}

% \begin{minted}{minted}
% \end{minted}


\end{document}
